%%%%%%%%%%%%%%%%%%%%%%%%%%%%%%%%%%%%%%%%%%%%%%%%%%%%%%%%%%%%%%%%%%%%%%%%%%%%%%%%
% Medium Length Graduate Curriculum Vitae
% LaTeX Template
% Version 1.2 (3/28/15)
%
% This template has been downloaded from:
% http://www.LaTeXTemplates.com
%
% Original author:
% Rensselaer Polytechnic Institute 
% (http://www.rpi.edu/dept/arc/training/latex/resumes/)
%
% Modified by:
% Daniel L Marks <xleafr@gmail.com> 3/28/2015
%
% Important note:
% This template requires the res.cls file to be in the same directory as the
% .tex file. The res.cls file provides the resume style used for structuring the
% document.
%
%%%%%%%%%%%%%%%%%%%%%%%%%%%%%%%%%%%%%%%%%%%%%%%%%%%%%%%%%%%%%%%%%%%%%%%%%%%%%%%%

%-------------------------------------------------------------------------------
%	PACKAGES AND OTHER DOCUMENT CONFIGURATIONS
%-------------------------------------------------------------------------------

%%%%%%%%%%%%%%%%%%%%%%%%%%%%%%%%%%%%%%%%%%%%%%%%%%%%%%%%%%%%%%%%%%%%%%%%%%%%%%%%
% You can have multiple style options the legal options ones are:
%
%   centered:	the name and address are centered at the top of the page 
%				(default)
%
%   line:		the name is the left with a horizontal line then the address to
%				the right
%
%   overlapped:	the section titles overlap the body text (default)
%
%   margin:		the section titles are to the left of the body text
%		
%   11pt:		use 11 point fonts instead of 10 point fonts
%
%   12pt:		use 12 point fonts instead of 10 point fonts
%
%%%%%%%%%%%%%%%%%%%%%%%%%%%%%%%%%%%%%%%%%%%%%%%%%%%%%%%%%%%%%%%%%%%%%%%%%%%%%%%%
% \title Resume 1/11/17

\documentclass[margin]{res}  

% Default font is the helvetica postscript font
\usepackage{helvet}
\renewcommand{\familydefault}{\sfdefault}

% Increase text height
\textheight=700pt
\pdfpagewidth=\paperwidth
\pdfpageheight=\paperheight
\begin{document}

%-------------------------------------------------------------------------------
%	NAME AND ADDRESS SECTION
%-------------------------------------------------------------------------------
\name{Joe Rickard}

% Note that addresses can be used for other contact information:
% -phone numbers
% -email addresses
% -linked-in profile

\address{ \\180 south 31st street \\Boulder CO}
\address{ \\joerickard.io\\joe.s.rickard@gmail.com}

% Uncomment to add a third address
%\address{Address 3 line 1\\Address 3 line 2\\Address 3 line 3}
%-------------------------------------------------------------------------------

\begin{resume}

%-------------------------------------------------------------------------------
%	PROJECTS SECTION
%-------------------------------------------------------------------------------
\section{PROJECTS}
\par
\textbf{Dalvonic}: 
This project started as an entry for HackCU 2016. It is an application that uses NLTK to provide a user with visual opinion analysis. Data is taken off Twitter, related by user ID and hashtags, allowing us to display the opinions of a set of users on one or more topics. This was written in Python using the Flask framework; CSS and JavaScript make up the front-end.
\par
\textbf{Linear Algebra Library}:
I've taken the time to write a C++ library to complete linear algebra functions for data analysis. While some already exist, I used this as an opportunity to further cement my understanding of the maths. With this Library I can take CSV inputs from a data set and end with node coordinates for use with the clustering algorithm of your choice. This allows a user to more easily visualize data similarity and distribution with a front end library such as D3. 
\par
\textbf{Neural Network}: 
I've worked with the N.E.A.T. genetic machine learning algorithm, attempting to optimize energy usage in heating a building. This work was done using C++. My models never reached a level of complexity where the results were usable, but the experience was valuable nonetheless. I am still interested in this problem, and intend to continue development on environmental control algorithms.
\par
\textbf{SQL Query Work}: 
I've done freelance SQL Query production for a Boulder start-up aiding their new version release. This included writing new queries, updating old queries, and optimizing much of the existing code. This was done in mySQL.

%-------------------------------------------------------------------------------

%-------------------------------------------------------------------------------
%	COMPUTER SKILLS SECTION
%-------------------------------------------------------------------------------
\section{COMPUTER\\SKILLS}

\textbf{Languages}: C, C++, C\#, Python, Bash, SQL, x86 Assembly
\\
\textbf{Github}: /joerickard
\\
%\textbf{Operating Systems}: 
%Unix, Linux, FreeBSD, Mac OSX, Windows, Android.
%-------------------------------------------------------------------------------

%-------------------------------------------------------------------------------
%	EXPERIENCE SECTION
%-------------------------------------------------------------------------------
% Modify the format of each position
\begin{format}
\title{l}\employer{r}\\
\dates{l}\location{r}\\
\body
\end{format}
%-------------------------------------------------------------------------------
\section{EXPERIENCE}

\employer{The Trade Desk}
\location{Boulder CO}
\dates{5/17-8/17}
\title{\textbf{Front-End Performance Work}}
\begin{position}
At the trade desk I worked on their embedded pixel for customers sites, ensuring reasonably fast load times regardless of partner server outages.
\end{position}

\employer{Toys2Life}
\location{Boulder CO}
\dates{10/16-6/17}
\title{\textbf{BLE and Language Model development}}
\begin{position}
I spent time working on an existing C\# code base. This involved significant re-factoring and creation of new functionality. The product involves custom firmware for the BLE stack, leveraged to gather relative location data of discrete nodes. On top of this I developed language models to give nodes the ability to have contextually driven conversations with each other. I also developed a graphical UI for internal content creation, this was also in C\#.
\end{position}

\employer{Lab for Atmospheric and Space Physics}
\location{Boulder CO}
\dates{02/15-10/16}
\title{\textbf{IT Administrator}}
\begin{position}
I worked to manage the in-house servers and maintain user access. This included regular backups, building new servers, migrating data between hardware, and interaction with a large VMware stack. The server-room housed racks for both database storage/access and computation.
\end{position}

%-------------------------------------------------------------------------------
%	EDUCATION SECTION
%-------------------------------------------------------------------------------
\section{EDUCATION}
\textbf{University of Colorado}, Boulder, CO\\
{\sl Seeking Bachelors:} Computer Science and Mathematics, expected Dec 2018\hfill \\
%{\sl Notable Courses:} Performant Linear Algebra, Concurrent Programming, Data Analysis Algorithms\hfill
\end{resume}
\end{document}



